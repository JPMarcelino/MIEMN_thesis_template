% ====================================================================
% GLOSSARY & NOTATION
% ====================================================================
% This file defines all the acronyms and notation used in the thesis.
% It uses the 'glossaries' package.
%
% HOW TO USE:
% 1. Add your acronyms and notations in the sections below.
% 2. To use an entry in your text, use the \gls{label} command.
%    For example, to use the 'FPGA' acronym, write \gls{fpga}.
%    The first time you use it, it will print the full text and the
%    abbreviation in parentheses, e.g., "Field-Programmable Gate Array (FPGA)".
%    Subsequent uses will only print the abbreviation, e.g., "FPGA".
%
% For more details on the 'glossaries' package, see its documentation.
% ====================================================================


% -------------------------------------------------------------------
% ACRONYMS
% -------------------------------------------------------------------
% Define all your acronyms here.
%
% The 'type=acronym' key is important because we have defined a
% separate glossary for acronyms in thesis.tex.
%
% Format:
% \newacronym[type=acronym]{label}{abbreviation}{full text}
%
% - 'label' is a unique identifier for the entry (e.g., 'fpga').
% - 'abbreviation' is the shortened form (e.g., 'FPGA').
% - 'full text' is the complete name (e.g., 'Field-Programmable Gate Array').
% -------------------------------------------------------------------

\newacronym[type=acronym]{fct}{FCT}{Faculdade de Ciências e Tecnologia}
\newacronym[type=acronym]{unl}{UNL}{Universidade Nova de Lisboa}
\newacronym[type=acronym]{asic}{ASIC}{Application-Specific Integrated Circuit}
\newacronym[type=acronym]{fpga}{FPGA}{Field-Programmable Gate Array}
\newacronym[type=acronym]{hdl}{HDL}{Hardware Description Language}


% -------------------------------------------------------------------
% NOTATION
% -------------------------------------------------------------------
% Define all your mathematical and scientific notations here.
%
% The 'type=notation' key is important because we have defined a
% separate glossary for notation in thesis.tex.
%
% Format:
% \newglossaryentry{label}{
%     type=notation,
%     name={symbol},
%     description={meaning}
% }
%
% - 'label' is a unique identifier (e.g., 'speedoflight').
% - 'name' is the symbol as it should appear in the text (e.g., \(c\)).
% - 'description' is the explanation of the symbol.
% -------------------------------------------------------------------

\newglossaryentry{speedoflight}{
    type=notation,
    name={\(c\)},
    description={The speed of light in a vacuum, approximately \(3.00 \times 10^8\) m/s.}
}

\newglossaryentry{planck}{
    type=notation,
    name={\(h\)},
    description={Planck's constant, relating a photon's energy to its frequency.}
}

\newglossaryentry{gravity}{
    type=notation,
    name={\(G\)},
    description={The gravitational constant, an empirical physical constant involved in the calculation of gravitational effects.}
}

% ====================================================================
% PRINTING THE GLOSSARIES
% ====================================================================
% These commands generate the lists of acronyms and notation in your
% document. They are placed here for convenience, but they are
% actually called from the main thesis.tex file.
%
% The 'style' key uses the custom tabular styles defined in thesis.tex.
% ====================================================================

\printglossary[type=acronym, style=tabularabbr, title={List of Abbreviations}]
\cleardoublepage

\printglossary[type=notation, style=tabularnotation, title={List of Notation}]
\cleardoublepage
