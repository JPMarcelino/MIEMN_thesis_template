% -------------------------------------------------------------------
% ACRONYMS
% -------------------------------------------------------------------
% The 'type' key is not needed here because we are using \newacronym,
% which automatically assigns the entry to the default 'main' glossary
% unless we specify another one. In thesis.tex, the 'acronym' glossary
% is defined with \newglossary[alg]{acronym}{...}, so we should use
% the 'type' option to assign it correctly.
%
% Usage:
% \newacronym[type=acronym]{label}{abbreviation}{full text}
%
% Example:
% \newacronym[type=acronym]{em}{EM}{Electro-Motive}
% -------------------------------------------------------------------

\newacronym[type=acronym]{fct}{FCT}{Faculdade de Ciências e Tecnologia}
\newacronym[type=acronym]{unl}{UNL}{Universidade Nova de Lisboa}
\newacronym[type=acronym]{asic}{ASIC}{Application-Specific Integrated Circuit}
\newacronym[type=acronym]{fpga}{FPGA}{Field-Programmable Gate Array}
\newacronym[type=acronym]{hdl}{HDL}{Hardware Description Language}


% -------------------------------------------------------------------
% NOTATION
% -------------------------------------------------------------------
% The 'type' key specifies that this entry belongs to the 'notation' glossary.
%
% Usage:
% \newglossaryentry{label}{type=notation, name={symbol}, description={meaning}}
%
% Example:
% \newglossaryentry{alpha}{type=notation, name={\(\alpha\)}, description={The fine-structure constant}}
% -------------------------------------------------------------------

\newglossaryentry{speedoflight}{
    type=notation,
    name={\(c\)},
    description={The speed of light in a vacuum, approximately \(3.00 \times 10^8\) m/s.}
}

\newglossaryentry{planck}{
    type=notation,
    name={\(h\)},
    description={Planck's constant, relating a photon's energy to its frequency.}
}

\newglossaryentry{gravity}{
    type=notation,
    name={\(G\)},
    description={The gravitational constant, an empirical physical constant involved in the calculation of gravitational effects.}
}

% -------------------------------------------------------------------
% PRINTING THE GLOSSARIES
% -------------------------------------------------------------------
% The following commands will print the glossaries in the document.
% The 'style' key uses the custom tabular styles defined in thesis.tex.
% -------------------------------------------------------------------

\printglossary[type=acronym, style=tabularabbr, title={List of Abbreviations}]
\cleardoublepage

\printglossary[type=notation, style=tabularnotation, title={List of Notation}]
\cleardoublepage
